\documentclass{article}
\usepackage{graphicx}
\usepackage[utf8]{inputenc}
\usepackage{listings}
\usepackage{hyperref}
\usepackage{color,soul}
\usepackage[a4paper, total={6in, 8in}]{geometry}
\title{Progetto Arithmetic Coding - LZ77}
\author{Alice Mariotti Nesurini - Enrico Bottani, I2AC}
\date{}

\begin{document}
\maketitle

\section{Struttura codice}

\subsection{Utilizzo}
\null
Compilazione
\begin{lstlisting}
gcc src/main.c src/lz77/lz77.c src/lz77/lz77.h src/tools/file/file.c 
src/tools/file/file.h src/tools/charcb/cb/cibuff.c src/tools/charcb/cb/cibuff.h 
src/tools/cmdef.h src/tools/bytestr/bytestr.c src/tools/bytestr/bytestr.h 
src/tools/bitfile/bitfilewriter.h src/tools/bitfile/bitfilewriter.c 
src/tools/bitfile/bitfilereader.c src/tools/bitfile/bitfilereader.h 
src/lz77/dec/dec.c src/lz77/dec/dec.h src/lz77/cmp/enc.c 
src/lz77/cmp/enc.h src/lz77/dec/dlog.c src/lz77/cmp/clog.c src/lz77/cmp/clog.h 
src/lz77/dec/dlog.h src/tools/utility.c src/tools/utility.h src/tools/kmp/kmp.c 
src/tools/kmp/kmp.h src/tools/kmp/kmplog.c src/tools/kmp/kmplog.h 
src/tools/scharcb/scharcb.c src/tools/scharcb/scharcb.h src/ac/ac_encoding.h 
src/ac/ac_encoding.c src/ac/element.c src/ac/element.h src/ac/ac_decoding.c 
src/ac/ac_decoding.h
\end{lstlisting}
\leavevmode
Compressione\\
LZ77AC -c ../../LZ77AC/test/alice\\\\
\null
Decompressione\\
LZ77AC -d ../../LZ77AC/test/alice.press


\section{Tempi compressione/decompressione}
\subsection{Arithmetic Coding}
\null
\begin{tabular}{lllll}
	Nome  & Size originale & Size compresso & Compressione [s] & Decompressione [s]\\
	\hline
	Alice.txt & 163.8 [kB] & 95 [kB] & 1.57 & 1.35\\
	Immagine.tiff & 3.4 [MB] & 3.2 [MB] & 66.3 & 31.5\\
	32k\_ff & 32.8 [kB] & 1 [kB] & 0.11 & 0.09\\
	32k\_random & 32.8 [kB] & 33.8 [kB] & 0.59 & 0.36\\
	ff\_ff\_ff & 3 [B] & 1 [kB] & 0.00 & 0.00\\
\end{tabular}

%tabella enrico
\subsection{LZ77}
\null
\begin{tabular}{lllll}
	Nome  & Size originale & Size compresso & Compressione [s] & Decompressione [s]\\
	\hline
	Alice.txt & 163.8 [kB] & 122.7 [kB] & 1.24 & 0.00\\
	Immagine.tiff & 3.4 [MB] & 4.9 [MB] & 45.92 & 0.19\\
	32k\_ff & 32.8 [kB] & 6.1 [kB] & 0.00 & 0.00\\
	32k\_random & 32.8 [kB] & 48.4 [kB] & 0.46 & 0.00\\
	ff\_ff\_ff & 3 [B] & 4 [B] & 0.00 & 0.00\\
\end{tabular}

%tabella insieme
\subsection{Algoritmi combinati}
\null
\begin{tabular}{lllll}
	Nome  & Size originale & Size compresso & Compressione [s] & Decompressione [s]\\
	\hline
	Alice.txt & 163.8 [kB] & 121.8 [kB] & 3.43 & 1.38\\
	Immagine.tiff & 3.4 [MB] & 4.8 [MB] & 127 & 49.7\\
	32k\_ff & 32.8 [kB] & 1.4 [kB] & 0.05 & 0.04\\
	32k\_random & 32.8 [kB] & 44 [kB] & 1.23 & 0.51\\
	ff\_ff\_ff & 3 [B] & 1 [kB] & 0.00 & 0.00\\
\end{tabular}

\section{Problemi}

Problemi sulla memoria, per risolvere questi problemi, che spesso erano allocazioni non ripulite correttamente, è stato molto utile valgrind (su linux) per debug.
\\\\
Il passaggio del numero totale di caratteri codificati e dell'array delle frequenze (per l'arithmetic coding) è stata un'aggiunta relativamente recente, e con questa aggiunta la dimensione dei file compressi aumenta notevolmente, ma visto che non è stato sviluppato un algoritmo ac addattivo, nel file di output questi dati devono essere presenti, essendo che il compressore non ha modo di avere queste informazioni in altri modi.
\\\\
Per migliorare i tempi di compressione abbiamo usato Instrument time profiling (su mac) e callgrind con KChacegrind su linux. Ci è stato utile per trovare quali funzioni impiegavano più tempo e applicare delle migliorie a tali funzioni e passaggi.

\section{Sviluppi futuri}

\subsection{Consistenza dati}
\hl{Sarebbe stato opportuno applicare l'arithmetic coding sul dizionario}
\hl{\textbf{Enrico}}

\end{document}