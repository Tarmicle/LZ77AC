\documentclass{article}
\usepackage{graphicx}
\usepackage[utf8]{inputenc}
\usepackage{listings}
\usepackage{hyperref}
\usepackage{color,soul}
\usepackage[a4paper, total={6in, 8in}]{geometry}
\title{Progetto Arithmetic Coding - LZ77}
\author{Alice Mariotti Nesurini - Enrico Bottani, I2AC}
\date{}

\begin{document}
\maketitle

\section{Struttura codice}

\subsection{Utilizzo}
\null
Compilazione
\begin{lstlisting}
gcc src/main.c src/lz77/lz77.c src/lz77/lz77.h src/tools/file/file.c 
src/tools/file/file.h src/tools/charcb/cb/cibuff.c src/tools/charcb/cb/cibuff.h 
src/tools/cmdef.h src/tools/bytestr/bytestr.c src/tools/bytestr/bytestr.h 
src/tools/bitfile/bitfilewriter.h src/tools/bitfile/bitfilewriter.c 
src/tools/bitfile/bitfilereader.c src/tools/bitfile/bitfilereader.h 
src/lz77/dec/dec.c src/lz77/dec/dec.h src/lz77/cmp/enc.c 
src/lz77/cmp/enc.h src/lz77/dec/dlog.c src/lz77/cmp/clog.c src/lz77/cmp/clog.h 
src/lz77/dec/dlog.h src/tools/utility.c src/tools/utility.h src/tools/kmp/kmp.c 
src/tools/kmp/kmp.h src/tools/kmp/kmplog.c src/tools/kmp/kmplog.h 
src/tools/scharcb/scharcb.c src/tools/scharcb/scharcb.h src/ac/ac_encoding.h 
src/ac/ac_encoding.c src/ac/element.c src/ac/element.h src/ac/ac_decoding.c 
src/ac/ac_decoding.h
\end{lstlisting}
\leavevmode
Compressione\\
LZ77AC -c ../../LZ77AC/test/alice\\\\
\null
Decompressione\\
LZ77AC -d ../../LZ77AC/test/alice.press


\section{Tempi compressione/decompressione}
\subsection{Arithmetic Coding}
\null
\begin{tabular}{lllll}
	Nome  & Size originale & Size compresso & Compressione [s] & Decompressione [s]\\
	\hline
	Alice.txt & 163.8 [kB] & 95 [kB] & 1.576 & 1.351\\
	Immagine.tiff & 3.4 [MB] & 3.2 [MB] & 66.32 & 31.51\\
	32k\_ff & 32.8 [kB] & 1 [kB] & 0.11 & 0.091\\
	32k\_random & 32.8 [kB] & 33.8 [kB] & 0.592 & 0.365\\
	ff\_ff\_ff & 3 [B] & 1 [kB] & 0.001 & 0.000\\
\end{tabular}

%tabella enrico
\subsection{LZ77}
\null
\begin{tabular}{lllll}
	Nome  & Size originale & Size compresso & Compressione [s] & Decompressione [s]\\
	\hline
	Alice.txt & 163.8 [kB] & 122.7 [kB] & 1.241 & 0.001\\
	Immagine.tiff & 3.4 [MB] & 4.9 [MB] & 45.92 & 0.190\\
	32k\_ff & 32.8 [kB] & 6.1 [kB] & 0.001 & 0.000\\
	32k\_random & 32.8 [kB] & 48.4 [kB] & 0.463 & 0.001\\
	ff\_ff\_ff & 3 [B] & 4 [B] & 0.000 & 0.000\\
\end{tabular}

%tabella insieme
\subsection{Algoritmi combinati}
\null
\begin{tabular}{lllll}
	Nome  & Size originale & Size compresso & Compressione [s] & Decompressione [s]\\
	\hline
	Alice.txt & 163.8 [kB] & 121.8 [kB] & 3.209 & 1.270\\
	Immagine.tiff & 3.4 [MB] & 4.8 [MB] & 124.0 & 49.71\\
	32k\_ff & 32.8 [kB] & 1.4 [kB] & 0.070 & 0.042\\
	32k\_random & 32.8 [kB] & 44 [kB] & 1.263 & 0.510\\
	ff\_ff\_ff & 3 [B] & 1 [kB] & 0.000 & 0.000\\
\end{tabular}

\section{Problemi}

Problemi sulla memoria, per risolvere questi problemi, che spesso erano allocazioni non ripulite correttamente, è stato molto utile valgrind (su linux) per debug.
\\\\
Il passaggio del numero totale di caratteri codificati e dell'array delle frequenze (per l'arithmetic coding) è stata un'aggiunta relativamente recente, e con questa aggiunta la dimensione dei file compressi aumenta notevolmente, ma visto che non è stato sviluppato un algoritmo ac addattivo, nel file di output questi dati devono essere presenti, essendo che il compressore non ha modo di avere queste informazioni in altri modi.
\\\\
Per migliorare i tempi di compressione abbiamo usato Instrument time profiling (su mac) e callgrind con KChacegrind su linux. Ci è stato utile per trovare quali funzioni impiegavano più tempo e applicare delle migliorie a tali funzioni e passaggi.

\section{Sviluppi futuri}

\subsection{Consistenza dati}

Al momento il nostro programma comprime prima con l'algoritmo LZ77 ed in seguito applica l'Arithmetic Coding sul risultato della prima codifica. Questo viene fatto leggendo il file compresso da LZ77 carattere per carattere incurante del significato dei bit processati. Una codifica sicuramente più efficiente avrebbe creato tabelle di frequenza separate per il dictionary, l'offset e il nuovo carattere mantenendo inoltre la consistenza dei dati passati. L'Arithmetic Coding leggendo infatti ogni 8 bit senza alcun tipo di parsing o cura del vaore di quest'ultima ha come come risultato di leggere caratteri molto casuali e quindi difficili da comprimere per un algoritmo entropico.
Per motivi di tempo non è stato possibile implementare questa importante feature.

\end{document}